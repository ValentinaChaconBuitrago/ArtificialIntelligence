% $ Id: introduction.tex  $
% !TEX root = main.tex

%%
\section{Introduction}
\label{sec:introduction}

According to encyclopedia Britannica, \ac{AI} is understood as the ability of a computer or 
computer-controlled machine to perform tasks associated with intelligent beings. The idea of 
\ac{AI} spawns from the work on \textit{artificial neurons} and the later development of reasoning 
programming languages (\eg PROLOG). \ac{AI} only began gaining proper recognition in the 
1980's, with the rise of expert systems~\cite{russel09}.  Recently, there has been a revival of 
\ac{AI}, partly due to the great amount of data available, collected through different devices, the 
great algorithmic development, and the ever growing computing power (\ie \ac{HPC} and Cloud). 
Different challenges and learning goals have risen in the development of \ac{AI} 
projects~\cite{russel09}. As a matter of fact, it is no longer possible to refer to \ac{AI} as a whole. 
Rather, it is necessary to take into account the type of \emph{reasoning}, \emph{learning}, or 
\emph{model} used by the machine. As is, there are two major categories in which \ac{AI} 
approaches can be classified:
\begin{enumerate*}[label=(\arabic*)]
\item \authorcomment[missing]{}{Enumerate the division of AI}, and
\item .
\end{enumerate*}

\ac{ML}~\cite{mitchell97} encompasses a set of algorithms that progressively improve their 
performance to execute a given set of tasks. \ac{ML} algorithms rely on data, using statistical 
analysis to learn or even predict their behavior based on previously unseen data. As a consequence, 
computers are now able to act and make decisions without programmers explicitly describing all 
tasks to perform. \ac{ML} algorithms can be used for a diverse set of applications such as shape 
recognition, patterns and speech recognition, effective web search, data classification, or medical 
diagnosis. \ac{ML} techniques are divided in the following XXX categories, as shown in 
\fref{fig:ai-categorization}.
\begin{description}
  \item[name\_technique:] explanation \authorcomment[missing]{}{Explain categories briefly, what is their main thing}
\end{description}

\begin{figure}[htbp]
  \centering
  \includegraphics[width=\textwidth]{images/ai-categorization}
  \caption{Categorization of \ac{AI}}
  \label{fig:ai-categorization}
\end{figure}

Given a particular problem, techniques belonging to different \ac{ML} categories can yield a solution. 
Even more, more than one technique may lead to the same result. 

The current landscape of software developments is characterized by systems requiring the processing 
of large volumes of data, and the exploitation of such data to improve the quality of the system or to 
offer specialized/personalized behavior. Examples of such systems can be found in the domains of 
\ac{IOT}~\cite{mattern10}, \acs{CPS}~\cite{holzl15}, or Smart Cities~\cite{zanella14}. \ac{AI} and 
\ac{ML} techniques seem to be appropriate for the development of such systems, given the large 
data to be processed, the need to adjust or evolve algorithms as more data is gathered, and the 
impossibility to provide programs for all possible future situations. 

Nonetheless, for many of these projects there are techniques that are more appropriate than others.
This paper offers an overview of the field of \ac{ML}, presenting the necessary background for its 
main representative techniques (\fref{sec:related}). That is, supervised \ac{ML}, and unsupervised 
\ac{ML}. For each of these models we present an example application, characterizing the main 
requirements to apply each technique (\fref{sec:validation}).


\endinput

Machine Learning has become a topic of interest because of its large field of application (deepen). Artificial intelligence is a broad area of knowledge comprised with several branches where each one has a different configuration and purpose. As a result, we consider one question: what field should a new user start working with if the goal is to classify information from a specific data set? Our work responds to this question based on empirical experimentation where we identified that approaching Machine Learning through supervised linear regression models allows users to understand the basic applications while fulfilling their objective. Moreover, due to available resources such as high level TensorFlow Estimator API and the column-oriented data analysis API Pandas learning doesn’t require long(subjective) before the user can start working on Machine Learning projects. 

