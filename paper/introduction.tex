% $ Id: introduction.tex  $
% !TEX root = main.tex

%%
\section{Introduction}
\label{sec:introduction}

According to encyclopedia Britannica, \ac{AI} is understood as the ability of a computer or 
computer-controlled machine to perform tasks associated with intelligent beings. The idea of 
\ac{AI} spawns from the work on \textit{artificial neurons} and the later development of reasoning 
programming languages (\eg PROLOG). \ac{AI} only began gaining proper recognition in the 
1980's, with the rise of expert systems~\cite{russel09}.  Recently, there has been a revival of 
\ac{AI}, partly due to the great amount of data available, collected through different devices, the 
great algorithmic development, and the ever growing computing power (\ie \ac{HPC} and Cloud). 
Different challenges and learning goals have risen in the development of \ac{AI} 
projects~\cite{russel09}. As a matter of fact, it is no longer possible to refer to \ac{AI} as a whole. 
Rather, it is necessary to take into account the type of \emph{reasoning}, \emph{learning}, or \emph{model} used by the machine. As is, there are two major categories in which \ac{AI} approaches can be classified:
\begin{enumerate*}[label=(\arabic*)]
\item \authorcomment[missing]{}{Enumerate the division of AI}, and
\item .
\end{enumerate*}

\authorcomment[missing]{}{machine learning definition}
\ac{ML}~\cite{watkins92} subdivision was studied. Therefore, one can say that \ac{ML} began as an initiative to solve AI problems by using data and learning from it. ML provides computers with the ability to make predictions with previously unseen data, this implies that computers are now able to act and make decisions without the intervention of a programmer. This broad field of study is rapidly growing due to its diverse applications such as shape, patterns and speech recognition, effective web search and medical diagnosis among others. ML comprises several techniques where each can solve specific task, and sometimes more than one technique may lead to the same results. ML can be distributed in several categories.

This paper will focus on Supervised Machine Learning, where models produce useful predictions of unseen data by combining inputs. \fref{fig:ai-categorization} \authorcomment[missing]{}{Explain categories, which one do we use and why}


\begin{enumerate}
 \item 1 [4]
 \item 2
\end{enumerate}


\begin{figure}[htbp]
  \centering
  \includegraphics[width=\textwidth]{images/ai-categorization}
  \caption{Categorization of \ac{AI} (taken from~\cite{})}
  \label{fig:ai-categorization}
\end{figure}

\endinput

Machine Learning has become a topic of interest because of its large field of application. Artificial intelligence is a broad area of knowledge comprised with several branches where each one has a different configuration and purpose. As a result, we consider one question: what field should a new user start working with if the goal is to classify information from a specific data set, what field is more appropriate for pattern recognition? Our work responds to this question based on empirical experimentation where we identified that approaching Machine Learning through supervised linear regression models allows users to understand the basic applications while fulfilling their data classification objective. Furthermore, we were able to recognize Neural Networks as a good source to classify patters in an effective and approachable way. Moreover, due to available resources such as high level TensorFlow Estimator API and the column-oriented data analysis API Pandas learning doesn’t require long before the user can start working on Machine Learning projects. 