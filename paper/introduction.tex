% $ Id: introduction.tex  $
% !TEX root = main.tex

%%
\section{Introduction}
\label{sec:introduction}

Britannica's definition, \ac{AI}~\cite{winston84} is understood as the ability of a computer or computer-controlled robot to perform tasks associated with intelligent beings. \ac{AI} has been around for several years, but it was only until the 1980’s that it began gaining proper recognition and more applications were developed.\footnote{\url{http://sitn.hms.harvard.edu/flash/2017/history-artificial-intelligence/}}  Moreover, AI has recently become a growing industry; partly because of the great amounts of data available, collected through different devices, and also because of great algorithmic development. However with great developments new challenges arise. As a result AI was divided into different categories:
\begin{enumerate*}[label=(\arabic*)]
\item \authorcomment[missing]{}{Enumerate the division of AI}, and
\item bkasjfasdashlahlj
\end{enumerate*}

\authorcomment[missing]{}{machine learning definition}
\ac{ML}~\cite{watkins92} subdivision was studied. Therefore, one can say (vocab) that\ac{ML} began as an initiative to solve AI problems by using data and learning from it. ML provides computers with the ability to make predictions with previously unseen data, this implies that computers are now able to act and make decisions without the intervention of a programmer. This broad field of study is rapidly growing due to its diverse applications such as shape, patterns and speech recognition, effective web search and medical diagnosis among others. ML comprises several techniques where each can solve specific task, and sometimes more than one technique may lead to the same results. ML can be distributed in several categories.

This paper will focus on Supervised Machine Learning, where models produce useful predictions of unseen data by combining inputs. \fref{fig:ai-categorization} \authorcomment[missing]{}{Explain categories, which one do we use and why}


\begin{enumerate}
 \item 1 [4]
 \item 2
\end{enumerate}


\begin{figure}[htbp]
  \centering
  \includegraphics[width=\textwidth]{images/ai-categorization}
  \caption{Categorization of \ac{AI} (taken from~\cite{})}
  \label{fig:ai-categorization}
\end{figure}

\endinput

Machine Learning has become a topic of interest because of its large field of application (deepen). Artificial intelligence is a broad area of knowledge comprised with several branches where each one has a different configuration and purpose. As a result, we consider one question: what field should a new user start working with if the goal is to classify information from a specific data set? Our work responds to this question based on empirical experimentation where we identified that approaching Machine Learning through supervised linear regression models allows users to understand the basic applications while fulfilling their objective. Moreover, due to available resources such as high level TensorFlow Estimator API and the column-oriented data analysis API Pandas learning doesn’t require long(subjective) before the user can start working on Machine Learning projects. 