% $ Id: introduction.tex  $
% !TEX root = main.tex

%%
\section{Introduction}
\label{sec:introduction}

According to Encyclopedia Britannica, \ac{AI} is understood as the ability of a computer or 
computer-controlled machine to perform tasks associated with intelligent beings. While the early 
idea of \ac{AI} spawns from the work on \textit{artificial neurons} and the later development of
reasoning programming languages (\eg Prolog). \ac{AI} only began gaining proper recognition 
in the 1980's, with the rise of expert systems~\cite{russel09}.  Recently, there has been a revival of 
\ac{AI}, partly due to the great amount of data available, collected through different devices, the 
great algorithmic development, and the ever growing computing power in domains as \ac{HPC} or 
Cloud. Different challenges and learning goals have risen in the development of \ac{AI} 
projects~\cite{russel09}. As a matter of fact, it is no longer possible to refer to \ac{AI} as a whole. 
Rather, it is necessary to take into account the type of \emph{reasoning}, \emph{learning}, or 
\emph{model} used by the machine. As is, there are two major categories in which \ac{AI} 
approaches can be classified:
\begin{enumerate*}[label=(\arabic*)]
\item applied systems, and
\item general systems.
\end{enumerate*}

Currently one of the most sought after question for \ac{AI} application is that of \emph{pattern 
recognition} from a data set. This is due to the requirements of current systems to process large 
volumes of data, and the exploitation of such data to improve the quality of the system, or to offer 
personalized behavior. Examples of such systems can be found in the domains of 
\ac{IOT}~\cite{mattern10}, \ac{CPS}~\cite{holzl15}, or Smart Cities~\cite{zanella14}. \ac{AI} and 
\ac{ML} techniques seem to be appropriate for the development of such systems, given the large 
data to be processed, the need to adjust or evolve algorithms as more data is gathered, and the 
impossibility to provide programs for all possible future situations. 
A particular technique from the general systems subcategory arises 
as a suitable candidate to solve these problems --that is, \ac{ML}~\cite{mitchell97}. 
\ac{ML} encompasses a set of algorithms that progressively improve their performance to execute 
a given set of tasks. \ac{ML} algorithms rely on data, using statistical analysis to learn or even predict 
their behavior based on previously unseen data. As a consequence, computers are now able to act 
and make decisions without programmers explicitly describing all tasks to perform.

\ac{ML} as a whole can solve a considerable amount of tasks that emerge from diverse backgrounds. 
In other words, \ac{ML} is powerful when dealing with real world problems~\cite{michalski13}, 
elevating the completeness of the solutions raised. 
The study on \ac{ML} began as an initiative to solve \ac{AI} problems by using data and learning from 
it. This broad field of study is rapidly growing due to its diverse applications such as shape, patterns 
and speech recognition, effective web search, and medical diagnosis. \ac{ML} comprises six main 
techniques to solve specific tasks, noting that multiple of such techniques may lead to the same 
results. \ac{ML} techniques are categorized as:
\begin{enumerate}
 \item Structured Prediction~\cite{taskar05}
 \item Ensemble Algorithms~\cite{dietterich00}
 \item Deep Learning~\cite{lecun15} 
 \item Unsupervised Learning~\cite{hastie09}
 \item Statistical Inference~\cite{robert14} 
 \item Probabilistic Learning~\cite{robert14}
\end{enumerate}

\begin{figure}[htbp]
  \centering
  \includegraphics[width=0.8\textwidth]{images/ai-categorization}
  \caption{Categorization of \acl{ML} techniques}
  \label{fig:ai-categorization}
\end{figure}

\fref{fig:ai-categorization} shows the exiting \ac{ML} approaches and their relations. Each of these 
models presents special characteristics that enhance results and provide users with tools for creating 
optimal models emerging in effective results.  Nevertheless, they each pursue different solutions for  
\ac{ML} problems, and sometimes they might not even work properly for all applications. 

This paper offers an overview of the field of \ac{ML}, presenting the necessary background for its 
main representative techniques, \ac{SML}, and \ac{NN} (\fref{sec:related}). 
\ac{SML} models produce useful predictions of unseen data by combining inputs, hence resulting in 
an appropriate way to classify information. \ac{NN} are considered as the leading method to achieve 
proper outcomes when working with patter recognition. \ac{NN} are trained based on data that has 
been labeled in advance. As a consequence, the learning process is based in analyzing training 
examples. Patterns are identified, if the current exploration  correlates with previously designated 
labels~\cite{mit17}. 

For each of these models we present an example application, characterizing the main 
requirements to apply each technique (\fref{sec:validation}). For the development of the applications 
we use the TensorFlow Estimator API and the column-oriented data analysis API Pandas.
We identified that supervised linear regression models allow users to understand the basic 
applications while fulfilling their data classification objective. Furthermore, we were able to recognize 
\ac{NN} as a good source to classify patters in an effective and approachable way. 


\endinput

