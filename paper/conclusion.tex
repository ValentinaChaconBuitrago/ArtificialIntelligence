% $  Id: conclusion.tex  $
% !TEX root = main.tex

%%
\section{Conclusion}
\label{sec:conclusion}

This paper offers a perspective on the wide subject of \ac{AI} and \ac{ML}. The purpose of the paper 
is disambiguate the purpose and use of \ac{ML} approaches for software development. In particular 
we focus on the question of what is the best technique for pattern recognition based on a given 
dataset. We explored the six existing approaches in \ac{ML}. The paper presents an empirical 
validation for the two approaches that exhibited the most promising results for the problem at hand, 
supervised learning and deep learning. The linear regression technique among the supervised 
learning approaches is identified as the most suitable technique for pattern recognition. Nonetheless, 
\acl{NN} also provide satisfactory results in patters classification. 
To conduct our experiments we use the TensorFlow Estimator API and the column-oriented data 
analysis API Pandas. We highlight the fact that the use of this frameworks eases the learning curve 
to \ac{ML}, and the start of a new \ac{ML} project.

As part of our future work, we would like to extend our empirical study in three directions. First we 
would like to cover all six sub-categories of \ac{ML}. Second, we would explore other application 
domains, different to pattern recognition. Finally, we would like to use the experiment to compare 
different frameworks realizing \ac{ML}, such as the Amazon AWS, or Artificial Neural Networks. 
As a final goal of this project we would like to build a characterization and a compendium of projects 
that are suitable for the use of \ac{ML}, and which one are not.

\endinput

